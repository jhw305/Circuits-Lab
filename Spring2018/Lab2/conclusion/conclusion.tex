\subsection{Part 1}
The NMOS exhibited normal $i_D$ versus $V_{GS}$ characteristics and applying a negative voltage to the bulk ($V_{SB} = 0.5$ \si{\volt}) effectively increased the threshold voltage $V_{tn}$, which is also the expected result. 
Overall, the NMOS performed well in this experiment and did not deviate from expected behavior.
\subsection{Part 2}
The $i_{D}$ versus $V_{DS}$ curves suggest that the transistor saturates near $V_{GS} - V_{tn}$ in each case.
This is consistent with both theory and simulation results.
\subsection{Part 3}
The $i_{D}$ versus $V_{SG}$ characteristics of the PMOS are as expected.
Furthermore, increasing $V_{SB}$ increases $|V_{tp}|$, which is in line with theory.
Enough informaton is available to determine the $\frac{W}{L}$ ratios for the transistors, which turn out to be $2.019$ for the NMOS and $1.740$ for the PMOS.
However, this is probably not terribly accurate since $k_{n}'$ and $k_{p}'$ are taken from the SPICE model.
The threshold voltage is considerably different from the SPICE model, so there is reason to believe that the $k$ parameters are as well.
These parameters should be determined through further experimentation before making a definitive statement about the $\frac{W}{L}$ ratio of the transistor.
\subsection{Part 4}
The $i_{D}$ versus $V_{SD}$ characteristics for the PMOS are not at all consistent with intuition nor theory.
One possibility is that the CD4007 chips have been damaged due to short circuits or other practical considerations, leading to the bizarre characteristics that make it act as a diode with a very high forward voltage.
However, this is a difficult argument to make since the $i_{D}$ versus $V_{SG}$ characteristics are in tact.
The measurement equipment or the circuit used to perform this analysis are likely the culprit.
