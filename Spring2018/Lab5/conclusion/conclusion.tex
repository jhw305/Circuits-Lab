The bias currents in the differential amplifier turn out to be slightly less in practice due to channel-length modulation effects.
Dropping $R_{REF}$ increases the bias current to the level at which it needs to be.
The differential and common-mode gain results align with the theoretical predictions.
The amplifier has a very low common-mode gain and a very high differential-mode gain, leading to a very high common-mode rejection ratio.
The output swing of the amplifier is limited by the transistors entering cutoff or saturation.
At either point, the output waveform clamps and distorts.
