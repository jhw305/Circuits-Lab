\documentclass[a4paper,10pt]{article}

\usepackage{spreadtab}
\usepackage{url}
\usepackage{siunitx}
\usepackage{graphicx}
\usepackage{placeins}
\usepackage{amsmath}

\begin{document}
	
The input impedance of the digital voltmeter is more accurately measured through the use of resistors with resistances that have comparable orders of magnitude to the input impedance of the voltmeter. For the digital voltmeter used in this experiment, the input resistance is in the M$\Omega$ range. Thus, the 1 M$\Omega$ resistor worked well to produce desired results. The 1 k$\Omega$ resistor, however, did not produce a good result because its resistance is insignificant compared to the input resistance of the digital voltmeter. The device tolerances in this case would be too significant compared to the ratio between the external resistor and the digital voltmeter's input resistance, thus introducing too much inaccuracy in calculations.
	
\end{document}
