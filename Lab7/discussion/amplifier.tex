The low-pass filtering characteristics of a common-emmiter amplifier are demonstrated. As expected, the gain drops as the operating frequency is increased. Thus, a cutoff frequency $f_c$ may be considered. The cutoff frequency is determined to be $150$\si{\kilo\hertz}. When applying the square wave input, the output waveforms differ depending on the duty cycle. At the cutoff frequency, the output waveform is different from what it is at a much lower frequency in the previous experiment due to delays in the BJT response. At a 20\% duty cycle, the output waveform slowly saturates to a maximum voltage until the input waveform experiences a rising edge. At this point, the output reverts back to ground. This process repeats itself at the input's falling edge. At a 50\% duty cycle, the input is low for too short of a time period to experience this voltage saturation. Thus, the output waveform is merely a small, short pulse at the rising edge of the input.
