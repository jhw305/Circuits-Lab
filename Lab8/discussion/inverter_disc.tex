The MOSFET exhibits the inverting capabilities predicted by theory. It also switches at a much faster rate than the bipolar-junction transistor (BJT). An equivalent BJT circuit would have rise and fall times as well as delays in the \si{\micro\second} range, whereas the MOSFET's are in the \si{\nano\second} range. Moreover, the MOSFET operates in saturation when the input is high and cutoff when the input is low, a desired characteristic of the circuit. Given the measured data, it is clear that the output voltage $V_{out}$ takes longer to transition from low to high than it does to transition from high to low. Furthermore, the MOSFET is unable to operate at high frequencies, such as $3$\si{\giga\hertz}, inasmuch as it is limited by its state transition delays.
