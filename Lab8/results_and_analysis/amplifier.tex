The inverter circuit constructed in Figure \ref{fig:mos_inverter} is also called a common-source amplifier and the amplification characteristics of the circuit is explored in this experiment by analyzing its frequency response.

For this portion of the experiment the MOSFET must always be operating in saturation mode, meaning the condition in Equation \ref{eq:sat_mode} must be satisfied. Because a sinusoidal wave is used as the input signal, a DC offset must be applied so that $V_{GD} < V_{Th} = 2$\si{\volt} for the entire amplitude range of the sinusoidal input. 

The amplifier's input signal is chosen to have $200$\si{\milli\volt}pp with a $2.5$\si{\volt}dc offset and a frequency of $10$\si{\kilo\hertz} because at this setting it is observed that the entire amplitude range of the input just lies within the saturation region. Clipping is observed in the output signal when the amplitude of the input signal is increased to $250$\si{\milli\volt}pp. This occurs because a portion of the positive cycle of the input signal now lies outside the saturation region and crosses over to the triode region. 

\FloatBarrier
\begin{figure}[h!]
	\centering
	\includegraphics[scale=0.25]{./images/amplifier_clamping.jpeg}
	\caption{Common-Source Amplifier $10$\si{\kilo\hertz} Distortions}
	\label{fig:amplifier_clipping}
\end{figure}
\FloatBarrier

Like the common-base amplifier observed in the previous experiment, the common-source amplifier is also a non-ideal amplifier due to capacitive effects. Consider the physical structure of the MOSFET, more specifically the nMOS:

\FloatBarrier
\begin{figure}[h!]
	\centering
	\includegraphics[scale=0.9]{./images/nmos_structure.JPG}
	\caption{Physical Structure of MOSFET}
	\label{fig:nmos_structure}
\end{figure}
\FloatBarrier

There is capacitance between the gate to source ($C_{gs}$) and gate to drain ($C_{gd}$) due to two reasons: the overlap of the metal gate electrode and the wells of n+ semiconductor material at the source and drain, and the close proximity of the metal gate electrode and the induced n-type channel. There is also junction capacitance between the n+ drain to p-type substrate body ($C_{db}$) and n+ source to p-type substrate body ($C_{sb}$) [1]. The equivalent circuit is given in the following model:

\FloatBarrier
\begin{figure}[h!]
	\centering
	\caption{MOSFET High Frequency Small Signal Model}
	\label{fig:mos_amp}
	\begin{circuitikz}
		\draw
		( 0 , 0 ) node[ nmos ] (my_nmos) {}
		
		% Gate
		(my_nmos.G) to [ short ] ++( -3 , 0 ) coordinate(g_out)
		(g_out) to [ sV , v<=$V_{in}$ ] ++( 0 , -3 ) coordinate(gnd_1)
		(gnd_1) node[ ground ] (my_gnd_1) {}
		
		% Drain
		(my_nmos.D) to ++(0,1) coordinate(r)
		(r) to [ R={$300\Omega$} ] ++( 0 , 2 ) coordinate(vcc)
		(vcc) to [ battery , v<=$V_{dd}\rightarrow5V$ ] ++( 3 , 0 ) coordinate(gnd_3)
		(gnd_3) node[ ground ] (my_gnd_3) {}
		
		% Source
		(my_nmos.S) to ++(0,-1) node[ ground ] (my_e_gnd) {}
		
		% Parasitic Caps
		(r) to [ C={$C_{db}$} ] ++( 2 , 0 ) coordinate(cdb)
		(cdb) node[ ground ] (my_gnd_4) {}
		(my_e_gnd) to [ C={$C_{sb}$} ] ++( 2 , 0 ) coordinate(csb)
		(csb) node[ ground ] (my_gnd_5) {}
		(r) to [ C, l_={$C_{gd}$} ] (g_out) 
		(my_e_gnd) to [ C={$C_{gs}$} ] (g_out) 
		
		;
	\end{circuitikz}
\end{figure}
\FloatBarrier

At low frequencies, the capacitors in the circuit above are essentially open circuits and thus the amplifier exhibits ideal behavior and maximum gain. However, when the frequency of the input is sufficiently high, the capacitors begin to short the path from the drain to ground, effectively decreasing the amplitude of the output voltage. Increasing the frequency of the input signal will drop the amplitude of the output signal further until $V_{out} = 0$\si{\volt}. This also causes the gain in \si{\decibel} of the amplifier to drop as gain is given by $20log(\frac{V_{out}}{V_{in}})$. 

At 10\si{\kilo\hertz}, the input voltage value of 200\si{\milli\volt}pp corresponds to an output voltage value of $3.3$\si{\volt}pp. $\frac{V_{out}}{V_{in}}$ at 10\si{\kilo\hertz} is calculated to be $16.5$ and a gain in \si{\decibel} of $24.3$\si{\decibel}. This gain value corresponds with the maximum gain because parasitic capacitances are still regarded as open circuits. The parasitic capacitances will then decrease the gain of the amplifier at high frequencies and a cutoff frequency, $f_c$, can be found when the gain in \si{\decibel} is a value that is $3$\si{\decibel} less than the maximum value or when the $\frac{V_{out}}{V_{in}}$ ratio is less than the maximum value by a factor of $\sqrt{2}$. The cutoff should then occur when $\frac{V_{out}}{V_{in}} = \frac{16.5}{\sqrt{2}} = 11.7$ which corresponds to a gain in \si{\decibel} of $21.3$\si{\decibel}. Using the condition above, $f_c$ is observed to be approximately $3.6$\si{\mega\hertz}.

\FloatBarrier
\begin{table}[h!]
	\centering
	\caption{Common-Source Amplifier Frequency Response}
	\label{tab:amplifier_freq}
	\csvautotabular{./tables/amplifier_freq.csv}
\end{table}
\FloatBarrier

A square wave with frequency of $f_c = 3.6$\si{\mega\hertz} is then set as the input signal to observe distortions in the output waveform. Two duty cycles for the square input are used for the observation: $20\%$ and $50\%$. The two duty cycles give the following waveforms:

\FloatBarrier
\begin{figure}[h!]
	\centering
	\includegraphics[scale=0.25]{./images/amplifier_20.jpeg}
	\caption{Common-Source Amplifier Square Wave Response $20\%$ Duty Cycle}
	\label{fig:amplifier_20}
\end{figure}
\FloatBarrier

\FloatBarrier
\begin{figure}[h!]
	\centering
	\includegraphics[scale=0.25]{./images/amplifier_50.jpeg}
	\caption{Common-Source Amplifier Square Wave Response $50\%$ Duty Cycle}
	\label{fig:amplifier_50}
\end{figure}
\FloatBarrier

For both the $20\%$ and $50\%$ duty cycle square inputs, the input and output signals have very noticeable distortions due to second-order effects. However, if the distortion is ignored, the output signal still resembles an inversion of the input signal. This is because at 3.6\si{\mega\hertz}, the period of the wave is $278$\si{\nano\second} which is still significantly higher than the delay times, rise time, and fall time found in the inverter experiment. This means that for both duty cycles, the output voltage still has sufficient time to fully develop.
