The following results are values indicated by the displays of the digital voltmeter and power supply:
	
\begin{table}[h!]
	\centering
	\caption{Voltmeter Measurements}
	\label{tab:voltmeter}
	% Voltmeter measurements
	% Voltage used: 5V
	\begin{spreadtab}{{tabular}{|c|c|c|}}
		\hline
		& @Voltage at R = 1k$\Omega$ [ V ] & @Voltage at R = 1M$\Omega$ [ V ] \\
		\hline
		@Voltmeter & 4.99 & 4.55 \\
		\hline
		@Power Supply & 4.98 & 4.98 \\ 
		\hline
	\end{spreadtab}
\end{table}
	
The setup with the 1 k$\Omega$ yields a voltmeter input voltage higher than the voltage of the power supply. By introducing these results to the voltage division equation in \ref{eq:vdvm}, a negative $R{DVM}$ value is computed.

\begin{equation}
\label{eq:vdvm_calc_1k}
\begin{gathered}
4.99 V = (4.98 V)(\frac{R_{DVM}}{R_{DVM} + 1 k\Omega})\\
4.99(R_{DVM} + 1 k\Omega) = 4.98 R_{DVM}\\
0.01 R_{DVM} = -4.99 k\Omega\\
R_{DVM} = -499 k\Omega
\end{gathered}
\end{equation}

The negative input resistance computed in \ref{eq:vdvm_calc_1k} is an impossible value so it is dropped.

By using \ref{eq:vdvm} for the 1 M$\Omega$ setup, the following input resistance is computed:

\begin{equation}
	\label{eq:vdvm_calc_1M}
	\begin{gathered}
		4.55 V = (4.98 V)(\frac{R_{DVM}}{R_{DVM} + 1 M\Omega})\\
		4.55(R_{DVM} + 1 M\Omega) = 4.98 R_{DVM}\\
		-0.43 R_{DVM} = -4.55 M\Omega\\
		R_{DVM} = 10.6 M\Omega
	\end{gathered}
\end{equation}
	
When compared to the 10M\si{\ohm} +/- 2\% input resistance indicated in the user manual of the digital multimeter, the following error is computed:

\begin{equation}
	\label{eq:vdvm_error}
	\begin{gathered}
		\frac{R_{DVM, measured} - R_{DVM, theoretical}}{R_{DVM, theoretical}}\\
	= \frac{10.6 M\Omega - 10 M\Omega}{10 M\Omega}\\
	= 5.8\%
	\end{gathered}
\end{equation}

Though this 5.8\% error is not within the manufacturer's predicted 2\% range, the value acquired is still quite close to the expected value. For resistances R closer to the expected value of R_{DVM}, the result is more accurate. It is conceivable that using a 10M$\Omega$ resistance R or some value close to it would yield a result fairly close to the manufacturer's specification.
