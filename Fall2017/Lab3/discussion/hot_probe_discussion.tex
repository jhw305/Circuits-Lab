While the hot probe technique yielded good results for the p-type silicon wafer, it did not produce discernible results for the n-type silicon wafers. This may be due to the size of the p-type silicon wafers that are tested. The areas of the p-type wafers that were subject to the hot probe technique are approximately one cubic centimeter. Due to the small size of the wafers, the contact with the hot probe may have caused uniform heating in the wafers rather than the desired temperature gradient. This then causes the wafer to essentially be at thermal equilibrium, thus no polarizing effect occurs and no significant voltage can be measured. Also, the higher mobility of electrons as charge carriers over holes may have played a role in the high fluctuation of the p-type semiconductors tested. Because of the free electrons' higher tendency to move in the material, the stability of the voltage measured across the ends of the p-type wafers may have suffered as a result. The hot probe technique in this experiment seems to be imprecise and crude in nature due to the instruments used and wafers tested. However, because the purpose of the experiment is not precision measurements but rather determination of type for semiconductor materials, the set up is adequate for that purpose.