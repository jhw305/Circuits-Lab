\FloatBarrier

\begin{figure}[h!]
	\centering
	\includegraphics[scale=0.75]{./images/circuit5.PNG}
	\caption{Common Source Amplifier with Complementary Load}
	\label{fig:circuit5}
\end{figure}

\FloatBarrier

\FloatBarrier

\begin{figure}[h!]
	\centering
	\includegraphics[scale=0.50]{./images/dc_sweep_vin.PNG}
	\caption{Voltage Transfer Characteristic}
	\label{fig:dc_sweep_vin}
\end{figure}

\FloatBarrier

\FloatBarrier

\begin{figure}[h!]
	\centering
	\includegraphics[scale=0.20]{./images/dc_sweep_vin_labeled.PNG}
	\caption{Voltage Transfer Characteristic with Regions of Operation}
	\label{fig:dc_sweep_vin_labeled}
\end{figure}

\FloatBarrier

% Find gain
The gain can be approximated by calculating the slope of the curve near the bias point, which is set to the point at which $V_{out} = \frac{V_{DD}}{2}$. For $R_1 = 1$\si{\kilo\ohm}, the gain is approximately $-45 \frac{V}{V}$. For $R_1 = 10$\si{\kilo\ohm}, the gain is approximately $-135 \frac{V}{V}$. For $R_1 = 100$\si{\kilo\ohm}, the gain is approximately $-422 \frac{V}{V}$. \\

\FloatBarrier

\begin{table}[h!]
	\centering
	\caption{Voltage Gain at Different $R_1$ Values}
	\label{tab:gain}
	\csvautotabular{./tables/gain.csv}
\end{table}

\FloatBarrier

% Pick R value for bias
Again, the best resistor value is the one that leads to the steepest descent in the saturation region. This is because the amplifier can prevent distortions by requiring smaller $V_{gs}$ variations for a particular output voltage, thereby improving its linearity. This occurs when the resistor is largest. Furthermore, the gain is highest when the resistor is largest. \uline{So, $R_1 = 100$\si{\kilo\ohm} is the best choice for a good voltage amplifier and inverter.} \\
