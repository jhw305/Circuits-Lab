\section{PMOS}
\subsection{Part A}
Using the circuit shown in figure \ref{fig:pmos_circuit} we measured $I_{sd}$ as we varied the drain voltage from within the range of $0 V \le V_{sd} \le 5 V$. 
The value of the gate voltage was set to $V_{sg} = 2.5V$. 
\\

\FloatBarrier

\begin{figure}[h!]
	\centering
	\includegraphics[scale=0.75]{../images/pmos_circuit}
	\caption{The PMOS transistor circuit used for our measurements.}
	\label{fig:pmos_circuit}
\end{figure}

\FloatBarrier
We can see from the $I_{sd}$ vs $V_{sd}$ curve in figure \ref{fig:pmos} that this PMOS transistor is in cutoff until $V_{sd} \approx 1.8 V$. 

\FloatBarrier

\begin{figure}[h!]
	\centering
	\includegraphics[scale=0.75]{../data/pmos.png}
	\caption{The resulting $I_{sd}$ vs $V_{sd}$ graph for $V_{sg}=2.5V$}
	\label{fig:pmos}
\end{figure}

\FloatBarrier
To turn on this transistor the gate voltage must be greater than the source voltage by at least the absolute value of the threshold voltage. 
This means that at $V_{sd} = 1.8 V$, $V_{sg} \ge |V_{tp}|$.
\\

To operate in triode mode the drain voltage must be greater than the gate voltage by at least the absolute value of the threshold voltage.
The transistor enters triode mode at $ V_{sd} \ge 1.8V$.
Given that our data did not show signs of entering saturation mode, we were unable to find the saturation edge.
\\

\subsection{Part B}
For part B we used the procedures from part A, but changed the gate voltage to $V_{sg} = 5 V$. 

\FloatBarrier

\begin{figure}[h!]
	\centering
	\includegraphics[scale=0.75]{../data/pmos_5v.png}
	\caption{The resulting $I_{sd}$ vs $V_{sd}$ graph for $V_{sg}=5V$}
	\label{fig:pmos_5v}
\end{figure}

\FloatBarrier
We see that the transistor operates in cutoff until $V_{sd} = 3.05 V$.
It then operates in triode mode for the rest of the values we tested up to $V_{sd}=6V$.
Given that our data did not show signs of entering saturation mode, we were unable to find the saturation edge.
