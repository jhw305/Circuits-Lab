\subsection{SPICE Simulation}
The D-latch design satisfies its behavioral description.
The only nonideality that occurs is the brief spiking behavior when $D$ is switched to $1$ and $C$ is switched to $0$.
So long as a sufficient amount of time, likely on the order of picoseconds at most, is given to allow the circuit's values to stabilize, this should not be a major concern.
The same transistor dimensions used in the SPICE simulation are used for the Microwind layout.
In SPICE, no physical issues arise from the transistor dimensions, but a true layout provides a more realistic simulation of the circuit's behavior when fabricated.
\subsection{Microwind Layout}
The final D-latch design outperforms all aspects of the specification by a decent amount.
However, the design can be made even better.
The area can decreased substantially by consolidating many of the PMOS and NMOS transistors.
For instance, the PMOSs in each NAND gate can be implemented using one p-diffusion region and two gates, sharing the same drain terminal, saving area on the chip.
Furthermore, speed and area can be saved by using only inverters and transmission gates in the design.
Signals would need to travel through fewer transistors, increasing speed and decreasing area.
